\documentclass[a4paper,10pt]{scrartcl}
\usepackage[utf8]{inputenc}
\usepackage{listings}
\title{Grundlegende Einführung in die Sprache und ihre Konzepte bzw. Rust is awesome}
\author{Philipp Schmid \and Andrea Matecsa}

%RUST Code Block
\usepackage{color}
\definecolor{mygreen}{rgb}{0,0.6,0}
\definecolor{mygray}{rgb}{0.5,0.5,0.5}
\definecolor{mymauve}{rgb}{0.58,0,0.82}
\lstset{ %
  backgroundcolor=\color{white},   % choose the background color; you must add \usepackage{color} or \usepackage{xcolor}; should come as last argument
  basicstyle=\footnotesize,        % the size of the fonts that are used for the code
  breakatwhitespace=false,         % sets if automatic breaks should only happen at whitespace
  breaklines=true,                 % sets automatic line breaking
  captionpos=b,                    % sets the caption-position to bottom
  commentstyle=\color{mygreen},    % comment style
  deletekeywords={...},            % if you want to delete keywords from the given language
  escapeinside={\%*}{*)},          % if you want to add LaTeX within your code
  extendedchars=true,              % lets you use non-ASCII characters; for 8-bits encodings only, does not work with UTF-8
  frame=single,	                   % adds a frame around the code
  keepspaces=true,                 % keeps spaces in text, useful for keeping indentation of code (possibly needs columns=flexible)
  keywordstyle=\color{blue},       % keyword style
  language=Octave,                 % the language of the code
  morekeywords={let,match,*,...},           % if you want to add more keywords to the set
  numbers=left,                    % where to put the line-numbers; possible values are (none, left, right)
  numbersep=5pt,                   % how far the line-numbers are from the code
  numberstyle=\tiny\color{mygray}, % the style that is used for the line-numbers
  rulecolor=\color{black},         % if not set, the frame-color may be changed on line-breaks within not-black text (e.g. comments (green here))
  showspaces=false,                % show spaces everywhere adding particular underscores; it overrides 'showstringspaces'
  showstringspaces=false,          % underline spaces within strings only
  showtabs=false,                  % show tabs within strings adding particular underscores
  stepnumber=1,                    % the step between two line-numbers. If it's 1, each line will be numbered
  stringstyle=\color{mymauve},     % string literal style
  tabsize=2,	                   % sets default tabsize to 2 spaces
  title=\lstname                   % show the filename of files included with \lstinputlisting; also try caption instead of title
}%kopiert von https://en.wikibooks.org/wiki/LaTeX/Source_Code_Listings#Supported_languages




\begin{document}
\maketitle
\tableofcontents
\section{Introduction} 
\subsection{Was ist Rust}Warum Rust? Was spricht für Rust?
erstellt von mozilla
\subsection{Besonderheiten von Rust}
-Focus auf Sicherheit
-ähnelt c und ocaml
\section{Pattern Matching} 
\subsection{was ist PM?}
Pattern Matching ist matchen auf ein bestimmtes Pattern
\subsection{Beispiel}
\begin{lstlisting}
let x=4;
match x{
1=>!println("x is 1")
2=>!println("x is 2")
3=>!println("x is 3")
4=>!println("x is 4")
_=>!println("x is anything else")
}
wie ocaml match a with x::xs->xs
\end{lstlisting}
\section{Mutability} Was ist mutability? wie wird es umgesetzt? vorteile nachteile?

\section{Higher Order Functions} Was sind HOF? Warum HOF? Beispiel, Gründe für HOF
breitere Wiederverwendung möglich
bekannte higher order functions bspw. fold 
Beispiel

\section{Comparison with other Languages}
Rust ist besser als alles andere
\subsection{Ocaml}
\subsection{andere}
\section{Discussion/Conclusion}
Rust ist supergeil

\begin{thebibliography}{9}
\bibitem{RustDocumentation}
  Official Rust Webpage,
  \emph{\LaTeX: Documentation about Rust},
  https://www.rust-lang.org/en-US/documentation.html
  accessed on the 10.4.2017
\end{thebibliography}

\end{document}